\documentclass[12pt]{article}
\usepackage{graphicx}
\usepackage{geometry}
\usepackage{multicol}

\makeatletter
% we use \prefix@<level> only if it is defined
\renewcommand{\@seccntformat}[1]{%
  \ifcsname prefix@#1\endcsname
    \csname prefix@#1\endcsname
  \else
    \csname the#1\endcsname\quad
  \fi}
% define \prefix@section
\newcommand\prefix@section{}
\newcommand\prefix@subsection{}
\newcommand\prefix@subsubsection{}
\makeatother

\title{Recepten}
\author{Tijmen van der Spijk}
\date{November 2022}

\begin{document}

\maketitle
\tableofcontents
\pagebreak

\section{Brood}
\pagebreak

\section{Ovenschotels}
\subsection{Witlofschotel met Tortillachips}
\subsubsection*{Ingrediënten}
\begin{multicols}{2}
    \begin{itemize}
        \item 1 blikje mais
        \item 1 bekertje creme fresh
        \item 2 blikken kidneybonen
        \item 1 zak naturel tortillachips
        \item 6 stronken witlof
        \item 4 tomaten
        \item 200 gram geraspte kaas.
    \end{itemize}
\end{multicols}

\subsubsection*{Instructies}
\begin{multicols}{2}
    \begin{enumerate}
        \item Verwarm de oven voor op 180 C
        \item Verwijder de kontjes van de witlof en snij ze in de lengte door.
        \item Snij ze een halve ringen van ongeveer 2 cm en kook alles beet gaar.
        \item Snij ondertussen de tomaat in (halve) ringen.
        \item Giet de witlof af, vet een ovenschaal in en verdeel de witlof over de ovenschaal.
        \item Verdeel de bonen en de mais erover en leg de tomaat plakje over het geheel heen.
        \item Verkruimel de chips in de zak en bedek de ovenschaal met een goed laag chips.
        \item Bestrooi het geheel met een royale laag kaas.
        \item Bak het geheel in de oven voor ongeveer 20-30 minuten totdat de kaas goud-bruin is.
        \item Serveer het geheel met creme fresh en chilisauce.
    \end{enumerate}
\end{multicols}
\pagebreak

\section{Pasta}
\subsection{Pasta Pesto}
\subsubsection*{Ingrediënten}
\begin{multicols}{2}
    \begin{itemize}
        \item 50 gram pijnboompitten
        \item 250 gram parmezaanse kaas
        \item 400 gram pasta naar keuze
        \item 2 teentjes knoflook
        \item snoeptomaatjes
        \item rucola
        \item verse basilicum
        \item olijfolie
    \end{itemize}
\end{multicols}

\subsubsection*{Instructies}
\begin{multicols}{2}
    \begin{enumerate}
        \item Kook de gekozen pasta al dente met een snufje zout.
        \item Rooster de pijnboompitten over laag vuur.
        \item Voeg de pijnboompitten toe in een keuken machine of een vijzel.
        \item Voeg de kaas, knoflook en de basilicum toe en zorg dat alles goed gemengd wordt.
        \item Voeg al mengende de olijfolie toe tot het geheel de gewenste dikheid heeft.
        \item Giet de pasta af en meng de pesto door de pasta.
        \item Serveer de pasta met wat snoeptomaatjes en wat rucola.
    \end{enumerate}
\end{multicols}
\pagebreak
\subsection{Pasta Caro}
\subsubsection*{Ingrediënten}
\begin{multicols}{2}
    \begin{itemize}
        \item 4 eieren
        \item 200 gram parmezaanse kaas
        \item 250 ml slagroom
        \item 250 gram spek
        \item 250 gram champignons
        \item 400 gram spaghetti
        \item oregano
    \end{itemize}
\end{multicols}

\subsubsection*{Instructies}
\begin{multicols}{2}
    \begin{enumerate}
        \item Kook de spaghetti al dente.
        \item Bak je spek in een klein beetje olijfolie krokant.
        \item Laat je spek uitlekken en leg deze apart op wat keuken papier.
        \item Bak de champignons zachtjes in dezelfde pan met wat peper zout en oregano.
        \item Klop ondertussen de eieren, de slagroom en de kaas tot het geheel ligt schuimt.
        \item Giet de spaghetti af en voeg de roommix, spek en champignons toe.
        \item Meng het geheel door en warm eventueel nog wat door.
    \end{enumerate}
\end{multicols}
\pagebreak
\subsection{Tagliatelle met Zalm-Roomsauce}
\subsubsection*{Ingrediënten}
\begin{multicols}{2}
    \begin{itemize}
        \item 1 bosje bosui
        \item 4 tomaten
        \item 200 ml creme fresh
        \item 250 gram gerookte zalm
        \item 400 gram tagliatelle
    \end{itemize}
\end{multicols}

\subsubsection*{Instructies}
\begin{multicols}{2}
    \begin{enumerate}
        \item Kook de tagliatelle al dente met een snufje zout. 
        \item Snij de tomaten in blokjes.
        \item Snij de bosui in kleine ringentjes en was dit even in een vergiet.
        \item Verhit wat boter in een pan en bak bosui erin.
        \item Voeg de creme fresh toe en breng het geheel aan de kook op laag vuur.
        \item Snij ondertussen de zalm in kleine snippers.
        \item Voeg de tomaat en de zalm toe en roer het geheel voorzichtig door.
        \item Breng het geheel nog op smaak met peper en zout.
    \end{enumerate}
\end{multicols}
\pagebreak
\subsection{Pasta met Makreel-Groene-Mosterd-sauce}
\subsubsection*{Ingrediënten}
\begin{multicols}{2}
    \begin{itemize}
        \item 400 gram pasta
        \item 1 ui
        \item 1 pot groene mosterd
        \item 3 teentjes knoflook
        \item 250 ml slagroom
        \item 500 gram makreel, gefileerd op het vel
        \item droge witte wijn
        \item visbouillonblokje
        \item sambal
        \item steranijs
        \item dragon
        \item bloem
    \end{itemize}
\end{multicols}

\subsubsection*{Instructies}
\begin{multicols}{2}
    \begin{enumerate}
        \item Kook de pasta naar keuze tot al dente
        \item Snipper het uitje en fruit deze rustig aan.
        \item Voeg de knoflook met een knoflookpers toe.
        \item Blus het geheel af met 300 ml wijn.
        \item Kook het geheel in tot 1/3.
        \item Voeg de steranijs, de slagroom en het bouillonblokje toe.
        \item Voeg ongeveer de helft van de groene mosterd toe en roer alles goed door.
        \item Voeg ook de dragon en een theelepel sambal toe en breng het geheel aan de kook.
        \item Verlaag het vuur en leg de stukjes makreel voorzichtig in de sauce.
        \item Voeg eventueel wat opgeklopte melk met bloem toe als de sauce te dun is.
        \item Voeg de sauce bij de pasta en mix alles goed door.
    \end{enumerate}
\end{multicols}
\pagebreak
\subsection{Lasagne}
\subsubsection*{Ingrediënten}
\begin{multicols}{2}
    \begin{itemize}
        \item 200 gram kaas
        \item 300 gram verse lasagnebladen
        \item 1 blikje tomatenpuree
        \item 2 blikken gepelde tomaten
        \item 2 zakken verse spinazie
        \item 4 teentjes knoflook
        \item oregano
        \item verse basilicum
    \end{itemize}
\end{multicols}

\subsubsection*{Instructies}
\begin{multicols}{2}
    \begin{enumerate}
        \item Verwarm de over voor op 180 °C.
        \item Snij de knoflook fijn.
        \item Breng de gepelde tomaten en de tomatenpuree langzaam aan de kook.
        \item Roerbak ondertussen de spinazie samen met wat knoflook in olijfolie in 4-6 delen.
        \item Breng de tomatensauce op smaak met wat oregano, basilicum, peper en zout.
        \item Laat de spinazie uitlekken, en vet de ovenschaal in met wat olijfolie.
        \item Bouw de lasagne op een een aantal lagen van sauce, spinazie en kaas.
        \item Verdeel de rest van de kaas over de bovenste laag.
        \item Bak het geheel in de oven totdat de kaas goud-bruin is.
    \end{enumerate}
\end{multicols}
\pagebreak

\section{Rijst}
\subsection{Pinto}

\subsubsection*{Ingrediënten}
\begin{multicols}{2}
    \begin{itemize}
        \item 2 uien
        \item 4 teentjes knoflook
        \item 2 chilipepers
        \item 2 paprika's
        \item 2 blikken kidneybonen
        \item 200-300 gram zilvervliesrijst
        \item 1-2 el komijnpoeder
        \item peper, zout en 1 bouillon blokje
    \end{itemize}
\end{multicols}

\subsubsection*{Instructies}
\begin{multicols}{2}
    \begin{enumerate}
        \item Breng de zilvervliesrijst aan de kook.
        \item Snipper ondertussen de uien, knoflook, chilipepers en de paprika's.
        \item Verhit in een diepe pan wat olie. Fruit hierin de uien circa 10 minuten tot goudbruin.
        \item Voeg de knoflook toe en bak de komijnpoeder mee voor ongeveer 1 minuut.
        \item Voeg de paprika en de chilipepers toe en roer alles goed door elkaar.
        \item Laat de alles even doorbakken en voeg daarna de bonen inclusief vocht bij de rest.
        \item Voeg het bouillon blokje en peper en zout toe naar smaak.
        \item Laat alles rustig inkoken totdat de rijst klaar is.
    \end{enumerate}
\end{multicols}
\pagebreak
\subsection{Paprika Schnitzel}
\subsubsection*{Ingrediënten}
\begin{multicols}{2}
    \begin{itemize}
        \item 400 ml crème fresh
        \item 2 grote uien
        \item 2 rode paprika's 
        \item 3 blikjes tomatenpuree
        \item schnitzels (1 pp)
        \item rijst (75-100 gram pp)
        \item tutti frutti
        \item paprikapoeder
        \item bloem
    \end{itemize}
\end{multicols}

\subsubsection*{Instructies}
\begin{multicols}{2}
    \begin{enumerate}
        \item Laat de tutti frutti weken in warm water.
        \item Zet deze op een laag vuurtje om alles rustig door te warmen.
        \item Breng de schnitzels op smaak met peper, zout en veel paprikapoeder.
        \item Snipper de uitjes en snij de paprika in blokjes.
        \item Fruit de uitjes.
        \item Haal de schnitzels ondertussen door wat bloem, en bak ze tot goud-bruin in boter.
        \item Voeg flink wat paprikapoeder toe bij de uien en roer alles goed door.
        \item Voeg de paprika en de tomatenpuree toe.
        \item Roer het geheel nog even door en laat het dan even sudderen voor ongeveer 10 minuten.
        \item Voeg de crème fresh toe en laat alles nog even een paar minuten sudderen.
    \end{enumerate}
\end{multicols}
\pagebreak
\subsection{Rijst met Pittige Worstjes}

\subsubsection*{Ingrediënten (3-4 personen)}
\begin{multicols}{2}
    \begin{itemize}
        \item 1 rode ui
        \item 1 chilipeper
        \item 1 rode paprika
        \item 2 teentjes knoflook
        \item 2 tomaten
        \item 4 lente uien
        \item pittige worstjes (1 pp)
        \item 300 gram rijst (risottorijst)
        \item 100 ml droge witte wijn
        \item 500 ml kippenbouillon
        \item bladpeterselie
        \item peper, zout en paprikapoeder
    \end{itemize}
\end{multicols}

\subsubsection*{Instructies}
\begin{multicols}{2}
    \begin{enumerate}
        \item Verhit olijfolie in een hoge brede pan.
        \item Snipper de rode ui redelijk fijn en fruit voor ongeveer 5 minuten tot ze zacht zijn.
        \item Snij ondertussen de chilipeper en de knoflook fijn en bak deze mee voor ongeveer 2 minuten.
        \item Ontdoe de worstjes van hun velletje en verdeel ze in kleine balletjes/blokjes.
        \item Doe de gebakken groentes even apart.
        \item Verhoog het vuur en voeg de worst toe aan de pan. Bak totdat ze ligt bruin zijn.
        \item Voeg 2 eetlepels paprikapoeder en peper en zout naar smaak toe.
        \item Voeg de gebakken groentes weer toe en bak voor een paar minuten.
        \item Voeg de rijst toe en roer alles door zodat de rijst alle smaken goed kan absorberen.
        \item Blus de pan af met de witte wijn en schraap de onderkant van de pan schoon.
        \item Voeg de kippenbouillon toe en breng het geheel aan de kook.
        \item Laat het geheel 15-20 minuten langzaam koken totdat de rijst gaar is.
        \item Snijdt ondertussen de tomaten in blokjes en de lente ui in ringetjes.
        \item Haal het geheel van het vuur.
        \item Voeg de tomaat en de lente ui toe en roer het geheel voorzichtig erdoor.
        \item Garneer het geheel met wat \break fijngesneden bladpeterselie.
    \end{enumerate}
\end{multicols}
\pagebreak
\subsection{Kip-Kerry met Rijst zonder Kip (TODO)}
\subsubsection*{Ingrediënten}
\begin{multicols}{2}
    \begin{itemize}
        \item ITEM 1
        \item ITEM 2
        \item ITEM 3
        \item ITEM 4
    \end{itemize}
\end{multicols}

\subsubsection*{Instructies}
\begin{multicols}{2}
    \begin{enumerate}
        \item INSTRUCTION 1
        \item INSTRUCTION 2
        \item INSTRUCTION 3
        \item INSTRUCTION 4
    \end{enumerate}
\end{multicols}
\pagebreak

\section{Salades}
\subsection{Geroosterde Bloemkool met Pita-Geitenkaas puntjes (TODO)}
\subsubsection*{Ingrediënten}
\begin{multicols}{2}
    \begin{itemize}
        \item 1 kg bloemkool
        \item 50 gram geroosterde amandelen
        \item 200 gram geitenkaas
        \item 1 pot gegrilde paprika
        \item 200 ml turkse/bulgaarse yoghurt
        \item bieslook
    \end{itemize}
\end{multicols}

\subsubsection*{Instructies}
\begin{multicols}{2}
    \begin{enumerate}
        \item INSTRUCTION 1
        \item INSTRUCTION 2
        \item INSTRUCTION 3
        \item INSTRUCTION 4
    \end{enumerate}
\end{multicols}
\pagebreak
\subsection{Bloemkool Salade (TODO)}
\subsubsection*{Ingrediënten}
\begin{multicols}{2}
    \begin{itemize}
        \item ITEM 1
        \item ITEM 2
        \item ITEM 3
        \item ITEM 4
    \end{itemize}
\end{multicols}

\subsubsection*{Instructies}
\begin{multicols}{2}
    \begin{enumerate}
        \item INSTRUCTION 1
        \item INSTRUCTION 2
        \item INSTRUCTION 3
        \item INSTRUCTION 4
    \end{enumerate}
\end{multicols}
\pagebreak
\subsection{Pasta Salade (TODO)}
\subsubsection*{Ingrediënten}
\begin{multicols}{2}
    \begin{itemize}
        \item ITEM 1
        \item ITEM 2
        \item ITEM 3
        \item ITEM 4
    \end{itemize}
\end{multicols}

\subsubsection*{Instructies}
\begin{multicols}{2}
    \begin{enumerate}
        \item INSTRUCTION 1
        \item INSTRUCTION 2
        \item INSTRUCTION 3
        \item INSTRUCTION 4
    \end{enumerate}
\end{multicols}
\pagebreak

\section{Soep}
\subsection{Bietensoep met geitenkaas-croutons}
\subsubsection*{Ingrediënten}
\begin{multicols}{2}
    \begin{itemize}
        \item 1 prei
        \item 1 ongesneden wit brood
        \item 4 sjalotten
        \item 4 teentjes knoflook
        \item 200 gram oude geitenkaas
        \item 400 gram gekookte bietjes
        \item tijm
        \item bouillonblokjes
    \end{itemize}
\end{multicols}

\subsubsection*{Instructies}
\begin{multicols}{2}
    \begin{enumerate}
        \item Snipper de sjalot en snij de knoflook en de prei fijn.
        \item Fruit alles rustig aan samen met wat peper en zout.
        \item Snij ondertussen het brood in blokjes van ongeveer 1 cm breed.
        \item Voeg kokend water (1 liter), de bietjes en de bouillonblokjes toe.
        \item Laat alles even rustig doorkoken
        \item Meng alle broodblokjes in een grote schaal met de tijm, kaas en wat peper.
        \item Verdeel ze over een bakplaat en bak ze op 200 C voor ongeveer 15 minuten.
        \item Pureer ondertussen de soep mooi glad en breng deze op smaak met wat peper en zout.
    \end{enumerate}
\end{multicols}
\pagebreak
\subsection{Broccoli Soep in Broodkom}
\subsubsection*{Ingrediënten}

\textbf{Brood:}

\begin{multicols}{2}
    \begin{enumerate}
        \item 360 g warm water (30 °C)
        \item 3 gram instant gist
        \item 550 gram broodmeel
        \item 11 gram zout
        \item 100 gram gerijpte polish
        \begin{itemize}
            \item 55 gram warm water
            \item snufje instant gist
            \item 55 gram broodmeel
        \end{itemize}        
    \end{enumerate}
\end{multicols}

\textbf{Soep}:

\begin{multicols}{2}
    \begin{enumerate}
        \item 1 grote ui
        \item 1 kleine venkel
        \item 6 teentjes knoflook
        \item 1 grote peen
        \item 2 kleine aardappels (yukon gold)
        \item 2 grote broccoli's
        \item 50 gram bloem
        \item 1,5 liter kippenbouillon
        \item 250 gram slagroom
        \item 225 gram cheddar
    \end{enumerate}
\end{multicols}

\subsubsection*{Instructies}
\begin{multicols}{2}
    \textbf{Brood}
    \begin{enumerate}
        \item Mix alle ingrediënten in een grote kom met een spatel.
        \item Maak het mixen af met een natte hand door te knijpen te de draaien.
        \item Leg een theedoek over de kom en laat het de rijzen voor 30 minuten.
        \item Trek en vouw het deeg 8-10 keer om het deeg structuur te geven.
        \item Leg een theedoek over de kom en laat het de rijzen voor nog eens 30 minuten.
        \item Herhaal de trek en vouw techniek voor nog eens 5-8 keer.
        \item Leg een theedoek over de kom en laat het de rijzen voor 60 minuten.
        \item Strooi wat bloem over het deeg en een snijplank.
        \item Verdeel het deeg in 4 delen van ongeveer 250 gram.
        \item Sla het deeg plat om een deel van het gas te verwijderen.
        \item Vouw het deeg 7-8 keer in zichzelf.
        \item Draai de deeg bal om en rol deze door de palm van je hand tot een mooie deegbal.
        \item Leg de deegbal op bakpapier.
        \item Herhaal dit voor de ander 3 delen.
        \item Leg een theedoek over de 4 deegballen en laat ze nog eens rijzen voor 60-90 minuten.
        \item Het deeg in klaar met rijzen als je er licht in kan porren en ze dan weer langzaam terug springen.
        \item Snij het bakpapier in vieren.
        \item Sproei wat water over de 4 ongebakken broodjes en snij een kruis in de boven kant met een broodmes.
        \item Schuif de 4 broodjes met bakpapier en al op een bakstaal in de oven.
        \item Besproei ze nog met wat extra wat en zet er een grote aluminium foliepan over.
        \item Bak de broodjes bedekt voor 15 minuten.
        \item Haal de foliepan er voorzichtig af. PAS OP VOOR DE STOOM.
        \item Bak de broodjes nog voor ongeveer 20 minuten tot goud-bruin.
    \end{enumerate}    
\end{multicols}

\begin{multicols}{2}
    \textbf{Soep}
    \begin{enumerate}
        \item Snij alle groentes
        \begin{itemize}
            \item zeer fijn: ui en venkel
            \item medium: wortel, aardappels en de broccoli
        \end{itemize}
        \item Zet een grote soep pan om medium-hoog vuur en voeg de olijf olie toe.
        \item Voeg alle groentes toe met een flinke snuf zout.
        \item Bak voor ongeveer 8-10 minuten.
        \item Bak de bloem mee voor nog eens 2 minuten.
        \item Voeg de kippenbouillon en de slagroom toe.
        \item Breng het geheel aan de kook en verlaag dan het gas to medium.
        \item Controleer of de groentes gaar zijn, als dit het geval is draai het gas naar de laagste stand.
        \item Voeg een aantal scheppen (4-5) van de soep in een blender beker en pureer dit.
        \item Voeg dit weer bij de soep.
        \item Voeg de kaas en peper en zout naar smaak toe.
        \item Maak kommetjes van het gebakken brood en serveer de soep hierin.
        \item Garneer met wat peper en kaas en het verwijderde brood.
    \end{enumerate}    
\end{multicols}


\pagebreak

\end{document}