\subsection{Geroosterde Bloemkool met Pita-Geitenkaas puntjes}
\subsubsection*{Ingrediënten}
\begin{multicols}{2}
    \begin{itemize}
        \item 1 kg bloemkool
        \item 50 gram geroosterde amandelen
        \item 200 gram geitenkaas
        \item 1 pot gegrilde paprika
        \item 200 ml Turkse/Bulgaarse yoghurt
        \item pitabroodjes
        \item bieslook
        \item za'atar kruidenmix
    \end{itemize}
\end{multicols}

\subsubsection*{Instructies}
\begin{multicols}{2}
    \begin{enumerate}
        \item Verwarm de over voor op 200 C.
        \item Snij de bloemkool in kleine roosjes en doe ze in een grote kom.
        \item Strooi ongeveer 2 eetlepels za'atar over de bloemkool en mix het geheel met wat olijfolie.
        \item Snij de pitabroodjes in de lengte door midden en maak er daarna puntjes van (6-8).
        \item Verdeel de bloemkool over een of meerdere bakplaten.
        \item Voeg de pitapuntjes toe in dezelfde kom als die van de bloemkool.
        \item Besprenkel het geheel met wat olijfolie en voeg de geitenkaas in kleine brokjes toe.
        \item Mix alles voorzichtig en leg ook de pitapuntjes op een bakplaten.
        \item Bak alle platen voor 15-20 minuten. Hussel halverwege de platen en wissel ze van plek.
        \item Laat de geroosterde paprika uitlekken en snij deze in kleine blokjes.
        \item Voeg ongeveer 50 ml van de vocht van de paprika samen met 200 ml yoghurt en flink wat bieslook.
        \item Hussel de paprika door de bloemkool in een grote kom en besprenkel het geheel met een deel van de yoghurt-dressing.
        \item Serveer de pitapuntjes ernaast.
    \end{enumerate}
\end{multicols}