\subsection{Broccoli Soep in Broodkom}
\subsubsection*{Ingrediënten}

\textbf{Brood:}

\begin{multicols}{2}
    \begin{enumerate}
        \item 360 g warm water (30 °C)
        \item 3 gram instant gist
        \item 550 gram broodmeel
        \item 11 gram zout
        \item 100 gram gerijpte polish
        \begin{itemize}
            \item 55 gram warm water
            \item snufje instant gist
            \item 55 gram broodmeel
        \end{itemize}        
    \end{enumerate}
\end{multicols}

\textbf{Soep}:

\begin{multicols}{2}
    \begin{enumerate}
        \item 1 grote ui
        \item 1 kleine venkel
        \item 6 teentjes knoflook
        \item 1 grote peen
        \item 2 kleine aardappels (yukon gold)
        \item 2 grote broccoli's
        \item 50 gram bloem
        \item 1,5 liter kippenbouillon
        \item 250 gram slagroom
        \item 225 gram cheddar
    \end{enumerate}
\end{multicols}

\subsubsection*{Instructies}
\begin{multicols}{2}
    \textbf{Brood}
    \begin{enumerate}
        \item Mix alle ingrediënten in een grote kom met een spatel.
        \item Maak het mixen af met een natte hand door te knijpen te de draaien.
        \item Leg een theedoek over de kom en laat het de rijzen voor 30 minuten.
        \item Trek en vouw het deeg 8-10 keer om het deeg structuur te geven.
        \item Leg een theedoek over de kom en laat het de rijzen voor nog eens 30 minuten.
        \item Herhaal de trek en vouw techniek voor nog eens 5-8 keer.
        \item Leg een theedoek over de kom en laat het de rijzen voor 60 minuten.
        \item Strooi wat bloem over het deeg en een snijplank.
        \item Verdeel het deeg in 4 delen van ongeveer 250 gram.
        \item Sla het deeg plat om een deel van het gas te verwijderen.
        \item Vouw het deeg 7-8 keer in zichzelf.
        \item Draai de deeg bal om en rol deze door de palm van je hand tot een mooie deegbal.
        \item Leg de deegbal op bakpapier.
        \item Herhaal dit voor de ander 3 delen.
        \item Leg een theedoek over de 4 deegballen en laat ze nog eens rijzen voor 60-90 minuten.
        \item Het deeg in klaar met rijzen als je er licht in kan porren en ze dan weer langzaam terug springen.
        \item Snij het bakpapier in vieren.
        \item Sproei wat water over de 4 ongebakken broodjes en snij een kruis in de boven kant met een broodmes.
        \item Schuif de 4 broodjes met bakpapier en al op een bakstaal in de oven.
        \item Besproei ze nog met wat extra wat en zet er een grote aluminium foliepan over.
        \item Bak de broodjes bedekt voor 15 minuten.
        \item Haal de foliepan er voorzichtig af. PAS OP VOOR DE STOOM.
        \item Bak de broodjes nog voor ongeveer 20 minuten tot goud-bruin.
    \end{enumerate}    
\end{multicols}

\begin{multicols}{2}
    \textbf{Soep}
    \begin{enumerate}
        \item Snij alle groentes
        \begin{itemize}
            \item zeer fijn: ui en venkel
            \item medium: wortel, aardappels en de broccoli
        \end{itemize}
        \item Zet een grote soep pan om medium-hoog vuur en voeg de olijf olie toe.
        \item Voeg alle groentes toe met een flinke snuf zout.
        \item Bak voor ongeveer 8-10 minuten.
        \item Bak de bloem mee voor nog eens 2 minuten.
        \item Voeg de kippenbouillon en de slagroom toe.
        \item Breng het geheel aan de kook en verlaag dan het gas to medium.
        \item Controleer of de groentes gaar zijn, als dit het geval is draai het gas naar de laagste stand.
        \item Voeg een aantal scheppen (4-5) van de soep in een blender beker en pureer dit.
        \item Voeg dit weer bij de soep.
        \item Voeg de kaas en peper en zout naar smaak toe.
        \item Maak kommetjes van het gebakken brood en serveer de soep hierin.
        \item Garneer met wat peper en kaas en het verwijderde brood.
    \end{enumerate}    
\end{multicols}

